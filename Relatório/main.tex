\documentclass{article}
\usepackage[brazil]{babel}
\usepackage[utf8]{inputenc}
\usepackage{amssymb}

\title{Relatório Criptografia RSA}
\date{\today}
\author{Bruno M. Barbosa}

\begin{document}
\maketitle

\section{Introdução}
Este relatório é referente à segunda parte da avaliação para seleção de alunos do IC para bolsistas do LCCV em 2019, que consiste em desenvolver um módulo na linguagem Python que implementa a criptografia RSA. 

\section{Método utilizado}
O método utilizado está descrito em~\cite{Evaristo2012}.  Primeiro, calcular a \textbf{chave de codificação}, que consiste em escolher dois números primos, $p$ e $q$ e calcular o produto $n = p * q$. Depois, calcular $\Phi(n) = \Phi(p) * \Phi(q)$, onde $\Phi(n) = n - 1$. Daí, achar um número $c$, tal que o $mdc(\Phi(n), c) = 1$, ou seja, primos entre si. Encontrado tal número, o par $(n, c)$ é a chave de codificação.

A \textbf{chave de decodificação} é o par $(n, d)$, onde $d$ é o número que multiplicado por $c$ e somado com $\Phi(n)$ resulta em 1.

Após as chaves serem determinadas, o próximo passo é transformar a mensagem em números. Os caracteres da mensagem são transformados em seus equivalentes na tabela ASCII e somados com 100. Então, a cadeia de números é dividida em partições que são primas entre si com n,ou seja, partições $B_i$, tais que $mdc(B_i, n) = 1$, com $i \in \mathbb{Z}$. 

Daí, a \textbf{função de codificação} irá fazer $B_i^c mod n$, onde $mod$ representa a operação módulo, o resto da divisão euclidiana, para gerar a mensagem codificada (no livro, os resultados são separados por $~\#$). A \textbf{função de decodificação} irá pegar a mensagem codificada e realizar o mesmo cálculo de módulo, porém, o expoente será $d$.

A chave de codificação é a chave pública e a chave de decodificação é a chave privada, que são exigidos no problema. 

\section{Implementação}
A implementação consiste em uma classe em Python, chamada RSA, que possui os primos $p$ e $q$, as chaves de codificação e decodificação e as mensagens codificadas e decodificadas como atributos. Possui as classes \textit{setup}, \textit{encrypt} e \textit{decrypt}, como exigido, e algumas classes adicionais, para a realização de cálculos necessários para a criptoografia RSA, como o cálculo do mdc, feito pela função \textit{gcd}, uma função para gerar números primos, \textit{$prime_numbers_until_n$} e uma função para achar o parâmetro $d$ na chave de decodificação. A função \textit{decrypt} devolve uma mensagem de erro, caso a mensagem seja inválida.

\bibliographystyle{plain}
\bibliography{reference}
\end{document}
